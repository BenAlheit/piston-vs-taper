
%% Packages
\usepackage{algorithm}% http://ctan.org/pkg/algorithm
\usepackage{algpseudocode}% http://ctan.org/pkg/algorithmicx
\usepackage{amsmath} % gathered
\usepackage{amssymb} % mathbb
\usepackage{authblk}
\usepackage[doi=false,isbn=false,url=false,eprint=false, backend=bibtex, maxbibnames=99,firstinits=true]{biblatex}
\renewbibmacro{in:}{}
\usepackage{blindtext}
\usepackage{bm} % mathbb
\usepackage{color}
\usepackage{float}
\usepackage[font=footnotesize]{caption}
\captionsetup[figure]{font=footnotesize}
\usepackage[margin=0.8in,]{geometry} % Make margins 0.8in inch
\linespread{1.25} % Use 1.5 linespacing (\linespread{x} standard is 1.2, 1.2*x = 1.5 => x = 1.25)
\usepackage{graphicx} % Include images
\usepackage{graphics}
\usepackage{hyperref} % Use hpyerlinks
\usepackage{nomencl}
\makenomenclature
\usepackage{etoolbox}
\renewcommand\nomgroup[1]{%
	\item[\bfseries
	\ifstrequal{#1}{P}{Physics constants}{%
		\ifstrequal{#1}{N}{Number sets}{%
			\ifstrequal{#1}{O}{Other symbols}{}}}%
	]}
\usepackage{subcaption} 





%% Defs
\renewcommand\Affilfont{\fontsize{9}{10.8}\itshape}
\renewcommand{\d}{\text{d}}

\newcommand{\tr}{\text{tr}}
\newcommand{\der}[2]{\dfrac{\text{d} #1}{\text{d} #2}}
\newcommand{\pder}[2]{\dfrac{\partial #1}{\partial #2}}
\newcommand{\vavg}[1]{\left<#1\right>}
\newcommand{\sig}{\bm{\sigma}}
\newcommand{\bsig}{\bar{\bm{\sigma}}}
\newcommand{\hsig}{\hat{\bm{\sigma}}}
\newcommand{\eps}{\varepsilon}
\newcommand{\beps}{\bar{\bm{\varepsilon}}}
\newcommand{\heps}{\hat{\bm{\varepsilon}}}
\newcommand{\Eeff}{E_{\text{eff}}}
\newcommand{\red}[1]{{\color{red}#1}}
\newcommand{\abaqus}{{\sc Abaqus}}
\newcommand{\umat}{{\texttt{\sc umat}}}
\newcommand{\vumat}{{\texttt{\sc vumat}}}


\newcommand{\singlefig}[4]{
	\begin{figure}[!htb]
		\centering
		\includegraphics[width=#1\linewidth]{#2}
		\caption{#3}
		\label{fig:#4}
	\end{figure}
}

\newcommand{\doublefig}[4]{
	\begin{figure}[!htb]
		\centering
		\begin{subfigure}{0.48\textwidth}
			\centering 
			\includegraphics[width=\linewidth]{#1} 
			\caption{}
		\end{subfigure}\hfill
		\begin{subfigure}{0.48\textwidth}
			\includegraphics[width=\linewidth]{#2}
			\caption{}
		\end{subfigure}
		
		\caption{#3}
		\label{fig:#4}
	\end{figure}
}

\newcommand{\quadfig}[6]{
	\begin{figure}[!htb]
		\centering
		\begin{subfigure}{0.48\textwidth}
			\centering 
			\includegraphics[width=\linewidth]{#1} 
			\caption{}
		\end{subfigure}\hfil
		\begin{subfigure}{0.48\textwidth}
			\includegraphics[width=\linewidth]{#2}
			\caption{}
		\end{subfigure}
		
		\begin{subfigure}{0.48\textwidth}
			\centering 
			\includegraphics[width=\linewidth]{#3} 
			\caption{}
		\end{subfigure}\hfil
		\begin{subfigure}{0.48\textwidth}
			\includegraphics[width=\linewidth]{#4}
			\caption{}
		\end{subfigure}
		
		\caption{#5}
		\label{fig:#6}
	\end{figure}
}


\newcommand{\quinfig}[8]{
	\begin{figure}[!htb]
		\centering
		\begin{subfigure}{#8\textwidth}
			\centering 
			\includegraphics[width=\linewidth]{#1} 
			\caption{}
		\end{subfigure}\hfil
		\begin{subfigure}{#8\textwidth}
			\includegraphics[width=\linewidth]{#2}
			\caption{}
		\end{subfigure}
		
		\begin{subfigure}{#8\textwidth}
			\centering 
			\includegraphics[width=\linewidth]{#3} 
			\caption{}
		\end{subfigure}\hfil
		\begin{subfigure}{#8\textwidth}
			\includegraphics[width=\linewidth]{#4}
			\caption{}
		\end{subfigure}
		
		\begin{subfigure}{#8\textwidth}
			\includegraphics[width=\linewidth]{#5}
			\caption{}
		\end{subfigure}
		
		\caption{#6}
		\label{fig:#7}
	\end{figure}
}

\newcommand{\hexfig}[8]{
	\begin{figure}[!htb]
		\centering
		\begin{subfigure}{0.31\textwidth}
			\centering 
			\includegraphics[width=\linewidth]{#1} 
			\caption{}
		\end{subfigure}\hfil
		\begin{subfigure}{0.31\textwidth}
			\includegraphics[width=\linewidth]{#2}
			\caption{}
		\end{subfigure}\hfil
		\begin{subfigure}{0.31\textwidth}
			\includegraphics[width=\linewidth]{#3}
			\caption{}
		\end{subfigure}
		
		\begin{subfigure}{0.31\textwidth}
			\centering 
			\includegraphics[width=\linewidth]{#4} 
			\caption{}
		\end{subfigure}\hfil
		\begin{subfigure}{0.31\textwidth}
			\includegraphics[width=\linewidth]{#5}
			\caption{}
		\end{subfigure}\hfil
		\begin{subfigure}{0.31\textwidth}
			\includegraphics[width=\linewidth]{#6}
			\caption{}
		\end{subfigure}
		
		\caption{#7}
		\label{fig:#8}
	\end{figure}
}


\newcommand{\heptfig}[9]{
	\begin{figure}[!htb]
		\centering
		\begin{subfigure}{0.31\textwidth}
			\centering 
			\includegraphics[width=\linewidth]{#1} 
			\caption{}
		\end{subfigure}\hfil
		\begin{subfigure}{0.31\textwidth}
			\includegraphics[width=\linewidth]{#2}
			\caption{}
		\end{subfigure}\hfil
		\begin{subfigure}{0.31\textwidth}
			\includegraphics[width=\linewidth]{#3}
			\caption{}
		\end{subfigure}
		
		\begin{subfigure}{0.31\textwidth}
			\centering 
			\includegraphics[width=\linewidth]{#4} 
			\caption{}
		\end{subfigure}\hfil
		\begin{subfigure}{0.31\textwidth}
			\includegraphics[width=\linewidth]{#5}
			\caption{}
		\end{subfigure}\hfil
		\begin{subfigure}{0.31\textwidth}
			\includegraphics[width=\linewidth]{#6}
			\caption{}
		\end{subfigure}
		
		
		\begin{subfigure}{0.32\textwidth}
			\includegraphics[width=\linewidth]{#7}
			\caption{}
		\end{subfigure}
		\caption{#8}
		\label{fig:#9}
	\end{figure}
}


