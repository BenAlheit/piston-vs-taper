\documentclass{article}


%% Packages
\usepackage{algorithm}% http://ctan.org/pkg/algorithm
\usepackage{algpseudocode}% http://ctan.org/pkg/algorithmicx
\usepackage{amsmath} % gathered
\usepackage{amssymb} % mathbb
\usepackage{authblk}
\usepackage[doi=false,isbn=false,url=false,eprint=false, backend=bibtex, maxbibnames=99,firstinits=true]{biblatex}
\renewbibmacro{in:}{}
\usepackage{blindtext}
\usepackage{bm} % mathbb
\usepackage{color}
\usepackage{float}
\usepackage[font=footnotesize]{caption}
\captionsetup[figure]{font=footnotesize}
\usepackage[margin=0.8in,]{geometry} % Make margins 0.8in inch
\linespread{1.25} % Use 1.5 linespacing (\linespread{x} standard is 1.2, 1.2*x = 1.5 => x = 1.25)
\usepackage{graphicx} % Include images
\usepackage{graphics}
\usepackage{hyperref} % Use hpyerlinks
\usepackage{nomencl}
\makenomenclature
\usepackage{etoolbox}
\renewcommand\nomgroup[1]{%
	\item[\bfseries
	\ifstrequal{#1}{P}{Physics constants}{%
		\ifstrequal{#1}{N}{Number sets}{%
			\ifstrequal{#1}{O}{Other symbols}{}}}%
	]}
\usepackage{subcaption} 





%% Defs
\renewcommand\Affilfont{\fontsize{9}{10.8}\itshape}
\renewcommand{\d}{\text{d}}

\newcommand{\tr}{\text{tr}}
\newcommand{\der}[2]{\dfrac{\text{d} #1}{\text{d} #2}}
\newcommand{\pder}[2]{\dfrac{\partial #1}{\partial #2}}
\newcommand{\vavg}[1]{\left<#1\right>}
\newcommand{\sig}{\bm{\sigma}}
\newcommand{\bsig}{\bar{\bm{\sigma}}}
\newcommand{\hsig}{\hat{\bm{\sigma}}}
\newcommand{\eps}{\varepsilon}
\newcommand{\beps}{\bar{\bm{\varepsilon}}}
\newcommand{\heps}{\hat{\bm{\varepsilon}}}
\newcommand{\Eeff}{E_{\text{eff}}}
\newcommand{\red}[1]{{\color{red}#1}}
\newcommand{\abaqus}{{\sc Abaqus}}
\newcommand{\umat}{{\texttt{\sc umat}}}
\newcommand{\vumat}{{\texttt{\sc vumat}}}


\newcommand{\singlefig}[4]{
	\begin{figure}[!htb]
		\centering
		\includegraphics[width=#1\linewidth]{#2}
		\caption{#3}
		\label{fig:#4}
	\end{figure}
}

\newcommand{\doublefig}[4]{
	\begin{figure}[!htb]
		\centering
		\begin{subfigure}{0.48\textwidth}
			\centering 
			\includegraphics[width=\linewidth]{#1} 
			\caption{}
		\end{subfigure}\hfill
		\begin{subfigure}{0.48\textwidth}
			\includegraphics[width=\linewidth]{#2}
			\caption{}
		\end{subfigure}
		
		\caption{#3}
		\label{fig:#4}
	\end{figure}
}

\newcommand{\quadfig}[6]{
	\begin{figure}[!htb]
		\centering
		\begin{subfigure}{0.48\textwidth}
			\centering 
			\includegraphics[width=\linewidth]{#1} 
			\caption{}
		\end{subfigure}\hfil
		\begin{subfigure}{0.48\textwidth}
			\includegraphics[width=\linewidth]{#2}
			\caption{}
		\end{subfigure}
		
		\begin{subfigure}{0.48\textwidth}
			\centering 
			\includegraphics[width=\linewidth]{#3} 
			\caption{}
		\end{subfigure}\hfil
		\begin{subfigure}{0.48\textwidth}
			\includegraphics[width=\linewidth]{#4}
			\caption{}
		\end{subfigure}
		
		\caption{#5}
		\label{fig:#6}
	\end{figure}
}


\newcommand{\quinfig}[8]{
	\begin{figure}[!htb]
		\centering
		\begin{subfigure}{#8\textwidth}
			\centering 
			\includegraphics[width=\linewidth]{#1} 
			\caption{}
		\end{subfigure}\hfil
		\begin{subfigure}{#8\textwidth}
			\includegraphics[width=\linewidth]{#2}
			\caption{}
		\end{subfigure}
		
		\begin{subfigure}{#8\textwidth}
			\centering 
			\includegraphics[width=\linewidth]{#3} 
			\caption{}
		\end{subfigure}\hfil
		\begin{subfigure}{#8\textwidth}
			\includegraphics[width=\linewidth]{#4}
			\caption{}
		\end{subfigure}
		
		\begin{subfigure}{#8\textwidth}
			\includegraphics[width=\linewidth]{#5}
			\caption{}
		\end{subfigure}
		
		\caption{#6}
		\label{fig:#7}
	\end{figure}
}

\newcommand{\hexfig}[8]{
	\begin{figure}[!htb]
		\centering
		\begin{subfigure}{0.31\textwidth}
			\centering 
			\includegraphics[width=\linewidth]{#1} 
			\caption{}
		\end{subfigure}\hfil
		\begin{subfigure}{0.31\textwidth}
			\includegraphics[width=\linewidth]{#2}
			\caption{}
		\end{subfigure}\hfil
		\begin{subfigure}{0.31\textwidth}
			\includegraphics[width=\linewidth]{#3}
			\caption{}
		\end{subfigure}
		
		\begin{subfigure}{0.31\textwidth}
			\centering 
			\includegraphics[width=\linewidth]{#4} 
			\caption{}
		\end{subfigure}\hfil
		\begin{subfigure}{0.31\textwidth}
			\includegraphics[width=\linewidth]{#5}
			\caption{}
		\end{subfigure}\hfil
		\begin{subfigure}{0.31\textwidth}
			\includegraphics[width=\linewidth]{#6}
			\caption{}
		\end{subfigure}
		
		\caption{#7}
		\label{fig:#8}
	\end{figure}
}


\newcommand{\heptfig}[9]{
	\begin{figure}[!htb]
		\centering
		\begin{subfigure}{0.31\textwidth}
			\centering 
			\includegraphics[width=\linewidth]{#1} 
			\caption{}
		\end{subfigure}\hfil
		\begin{subfigure}{0.31\textwidth}
			\includegraphics[width=\linewidth]{#2}
			\caption{}
		\end{subfigure}\hfil
		\begin{subfigure}{0.31\textwidth}
			\includegraphics[width=\linewidth]{#3}
			\caption{}
		\end{subfigure}
		
		\begin{subfigure}{0.31\textwidth}
			\centering 
			\includegraphics[width=\linewidth]{#4} 
			\caption{}
		\end{subfigure}\hfil
		\begin{subfigure}{0.31\textwidth}
			\includegraphics[width=\linewidth]{#5}
			\caption{}
		\end{subfigure}\hfil
		\begin{subfigure}{0.31\textwidth}
			\includegraphics[width=\linewidth]{#6}
			\caption{}
		\end{subfigure}
		
		
		\begin{subfigure}{0.32\textwidth}
			\includegraphics[width=\linewidth]{#7}
			\caption{}
		\end{subfigure}
		\caption{#8}
		\label{fig:#9}
	\end{figure}
}



\graphicspath{{./inkscape-figures/}{./lit-figures/}} 
\bibliography{piston-vs-taper.bib}

\title{Development of a machine learning model to determine the forces on the piston in the pump-tube of a two-stage gas gun deforming due to a taper}
\author[1, 2]{Benjamin Alheit\thanks{alhben001@myuct.ac.za}}
\date{\today}

\affil[1]{Centre for Research in Computational \& Applied Mechanics, University of Cape Town, 7701 Rondebosch, South Africa}
\affil[2]{Department of Mechanical Engineering, University of Cape Town, 7701 Rondebosch, South Africa}

\begin{document}
\maketitle
\begin{abstract}
\red{	Abstract here.}
\end{abstract}
\red{
\section{TODO}
\begin{enumerate}
	\item Get output parameters
	\item Model automation
	\item Mesh optimization
\end{enumerate}
}

\section{Introduction}
=
\red{
\begin{itemize}
	\item We will use: ton, mm, s, N, MPa, N-mm
\end{itemize}
}

\singlefig{0.7}{units}{Consistent units for \abaqus. From https://www.researchgate.net/post/What-are-the-Abaqus-Units-in-Visualization}{units}

\subsection{Gas gun design}

\subsubsection{title}

\section{Finite element simulation}

\subsection{Material models}

\subsubsection{ABS}

\red{
\begin{enumerate}
	\item Yield stress: 48.26 MPa from https://peer.asee.org/tensile-comparison-of-polymer-specimens-produced-with-different-processes.pdf
	\item (Modelled as perfect plasticity)
	\item Poisson's ration: 0.35 from http://www.goodfellow.com/A/Polyacrylonitrile-butadiene-styrene.html
	\item Young's modulus: 2.1-2.4 GPa from http://www.goodfellow.com/A/Polyacrylonitrile-butadiene-styrene.html
	\item Density $1.05 \times 10 ^{-9}$ ton/mm$^3$ from http://www.goodfellow.com/A/Polyacrylonitrile-butadiene-styrene.html
\end{enumerate}
}

\subsubsection{Steel}

\red{
	\begin{enumerate}
		\item Poisson's ration: 0.3 from http://www.matweb.com/search/datasheet.aspx?bassnum=MS0001\&ckck=1
		\item Young's modulus: 200 GPa from http://www.matweb.com/search/datasheet.aspx?bassnum=MS0001\&ckck=1
		\item Density $8 \times 10 ^{-9}$ ton/mm$^3$ from http://www.matweb.com/search/datasheet.aspx?bassnum=MS0001\&ckck=1
	\end{enumerate}
}






\subsubsection{HDPE}
Linear elastoplastic
\begin{enumerate}
	\item Density -- $\rho_{c}=0.95\times 10^{-9}$,ton/mm\textsuperscript{3}
	\item Young's modulus -- $E_{c}=1000$\,MPa
	\item Poisson's ratio -- $\nu_{c}= 0.46$
	\item Yield stress -- $\sigma_{yc}=12.5$\,MPa
\end{enumerate}

\subsubsection{Aluminium}
\begin{enumerate}
	\item 
\end{enumerate}

\subsection{Single test simulation}

\subsection{Mesh optimization}

Parameters
\begin{enumerate}
	\item ratio of element expansion for piston
	\item ratio of element expansion for tube
	\item n elements
\end{enumerate}

Objective function
\begin{enumerate}
	\item Min elements
\end{enumerate}

Constraint
\begin{enumerate}
	\item Force
	\item Dissipation
\end{enumerate}

\section{Machine learning surrogate model}
\subsection{Feature engineering}

\red{
Predictive features:
\begin{enumerate}
	\item Coefficient of friction: $\mu$
	\item Taper angle: $\alpha$
	\item Velocity: $v$
	\item Distance between piston front and taper start: $x_{taper}$
	\item Pressure difference between piston front and back: $\Delta p$
	\item Piston length: $l_{p}$
	\item Piston density: $\rho_{p}$
	\item Accumulative plastic strain in the piston: $\gamma$ 
\end{enumerate}

Dependent variables:
\begin{enumerate}
	\item Axial force on piston due to taper: $F_{z}$
	\item Increment in accumulated plastic dissipation: $\Delta \gamma$ 
\end{enumerate}
}

\subsubsection{Dimensional analysis}

\begin{table}[!htb]
	\centering
	\caption{Fundamental dimensional units}
	\label{tab:dimentional-units}
	\begin{tabular}{c c c}
		\hline\hline
		Mass $[M]$ & Length $[L]$ & Time $[T]$\\
		\hline\hline
	\end{tabular}
\end{table}

\begin{table}[!htb]
	\centering
	\caption{Variables on which the problem depends \red{Check that units are correct}}
	\label{tab:dimentional units}
	\begin{tabular}{l p{7cm} c c}
		\hline
		Component & Description & Symbol & Units \\
		\hline \hline
		Cap & Axial contact force & $F_{c}$ & $[MLT^{-2}]$ \\
		&Rate of volume average plastic dissipation & $\vavg{\dot{\Phi}_{c}}$ & $[ML^{-1}T^{-3}]$\\
		&Volume average plastic dissipation & $\vavg{\Phi_{c}}$ & $[ML^{-1}T^{-2}]$\\
		& Pressure difference & $\Delta P$ & $[ML^{-1}T^{-2}]$\\
		& Velocity & $v$ & $[LT^{-1}]$\\
		&Distance of cap past taper & $x_{t}$ & $[L]$\\
		&Length past piston & $l_{c}$ & $[L]$\\
		&Diameter & $d_{c}$ & $[L]$\\
		&Density & $\rho_{c}$ & $[ML^{-3}]$\\
		&Young's modulus & $E_{c}$ & $[ML^{-1}T^{-2}]$\\
		&Poisson's ratio & $\nu_{c}$ & $[\bullet]$\\
		&Yield stress & $\sigma_{yc}$ & $[ML^{-1}T^{-2}]$\\
		\hline
		Piston&Length & $l_{p}$ & $[L]$\\
		&Diameter & $d_{p}$ & $[L]$\\
		&Density & $\rho_{p}$ & $[ML^{-3}]$\\
		&Young's modulus & $E_{p}$ & $[ML^{-1}T^{-2}]$\\
		&Poisson's ratio & $\nu_{p}$ & $[\bullet]$\\
		\hline
		Transition piece& Length & $l_{t}$ & $[L]$\\
		& Diameter change & $\Delta d_{t}$ & $[L]$\\
		&Density & $\rho_{t}$ & $[ML^{-3}]$\\
		&Young's modulus & $E_{t}$ & $[ML^{-1}T^{-2}]$\\
		&Poisson's ratio & $\nu_{t}$ & $[\bullet]$\\
		\hline
		Friction & Cap-on-steel CoF & $\mu_{cs}$ & $[\bullet]$\\	
		&Steel-on-steel CoF & $\mu_{ss}$ & $[\bullet]$\\
		&Aluminium-on-steel CoF & $\mu_{as}$ & $[\bullet]$\\	
	\end{tabular}
\end{table}

\begin{equation}
	\Pi = F_c^{k_{1}}\vavg{\dot{\Phi}_{c}}^{k_{2}}\Delta P^{k_{4}}v^{k_{5}}\vavg{\Phi_{c}}^{k_{3}}x_{t}^{k_{6}}
\end{equation}
\begin{equation}
	r=3
\end{equation}
\begin{equation}
	n=24
\end{equation}
\begin{equation}
	n_{\text{n-dim}} = n-r=21
\end{equation}

\begin{equation}
	\newcommand{\dimmatspace}{\,\,\,\,\,}
	\begin{matrix}
		& 
		\,\,\,\,\,\,_{F_c} \dimmatspace _{\vavg{\dot{\Phi}_{c}}} \dimmatspace _{\Delta P} \,\,\dimmatspace _{v} \,\, \dimmatspace _{\vavg{\Phi_{c}}} \dimmatspace _{x_{t}} \dimmatspace
		\\
		\begin{matrix}
			_M \\ _L \\ _T 
		\end{matrix} & \left[\begin{matrix}
			1  &  1 &  1 &  0 &  1 & 0\\ 
			1  & -1 &  -1 &  1 & -1 & 1\\ 
			-2 & -3 & -2 & -1 & -2 & 0\\ 
		\end{matrix}\right]
	\end{matrix} 	\left[\begin{matrix}
	k_{1} \\  k_{2} \\ k_{3} \\ k_{4} \\ k_{5} \\ k_{6} 
\end{matrix}\right] = \bm{0}\,.
\end{equation}
\red{
\begin{enumerate}
	\item Enter units as symbols in sympy
	\item Construct dimensional matrix
	\item Find solutions $k^{j}_i$ to dimensional matrix by $n-r$ values in $k$ to 0 (except for 1 which on sets to $1$).
	\item Automatically print out resulting dimensionless products table.
\end{enumerate}
}
\begin{table}
	\centering
	\caption{Non-dimensional parameters}
	\label{tab:non-dim}
	\begin{tabular}{c c c c c c c}
		& $F_c$ & $\vavg{\dot{\Phi}_{c}}$ & $\vavg{\Phi_{c}}$ & $\Delta P$ & $v$ & $x_{t}$ \\
		& $k_{1}$ & $k_{2}$ & $k_{3}$& $k_{4}$& $k_{5}$& $k_{6}$ \\
		
	\end{tabular}
\end{table}

\red{
Based on \cite{hutter2013continuum}:

Notes:
\begin{enumerate}
	\item 
\end{enumerate}
Direct quote from \cite{hutter2013continuum}
\begin{quote}
	``Dimensional analysis is a method with the aid of which one may for instance test a formula for dimensional correctness. It leads to a first understanding of the solution of a physical problem and yields a precise information about the number of variables that are necessary to describe it, a fact that is particularly important when experiments are being performed. Very often dimensional analysis reduces the number of variables upon which a physical problem was initially surmised to depend. If for instance the quantity $y$ depends upon $x_{1}, x_{2} , ... , x_{n}$, where all quantities have a certain physical dimension, then dimensional analysis shows that y can only depend upon certain products of powers of $x_{1}, x_{2} , ... , x_{n}$, a fact that corresponds regularly to a considerable reduction of the number of variables. Naturally then, experiments may more simply or more economically be performed than without knowledge of this fact.''
\end{quote}

\begin{quote}
	``The first step in a dimensional analysis consists in the listing of the parameters, which influence a physical problem. This step is very decisive. If too many variables are listed that may describe a physical problem, then the final equations will contain superfluous variables, if too few variables are introduced, incomplete equations may emerge, which results in incomplete equations or "more often" false inferences or the result can not be expressed in terms of dimensionally homogeneous functions.''
\end{quote}
}

\subsection{Experimental input parameters}
 
In order of importance:
\begin{enumerate}
	\item Initial velocity: $v_{0}$
	\item Pressure path: $p_{path}$
	\item Piston length: $l_{p}$
	\item Coefficient of friction: $\mu$
\end{enumerate}

\subsection{Experimental results}

\subsection{Model}

\section{Packaging of model for use in 1D code}

\subsection{PIP}

\subsection{Usage example}


	\printbibliography
\end{document}