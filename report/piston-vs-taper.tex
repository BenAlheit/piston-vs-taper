\documentclass{article}


%% Packages
\usepackage{algorithm}% http://ctan.org/pkg/algorithm
\usepackage{algpseudocode}% http://ctan.org/pkg/algorithmicx
\usepackage{amsmath} % gathered
\usepackage{amssymb} % mathbb
\usepackage{authblk}
\usepackage[doi=false,isbn=false,url=false,eprint=false, backend=bibtex, maxbibnames=99,firstinits=true]{biblatex}
\renewbibmacro{in:}{}
\usepackage{blindtext}
\usepackage{bm} % mathbb
\usepackage{color}
\usepackage{float}
\usepackage[font=footnotesize]{caption}
\captionsetup[figure]{font=footnotesize}
\usepackage[margin=0.8in,]{geometry} % Make margins 0.8in inch
\linespread{1.25} % Use 1.5 linespacing (\linespread{x} standard is 1.2, 1.2*x = 1.5 => x = 1.25)
\usepackage{graphicx} % Include images
\usepackage{graphics}
\usepackage{hyperref} % Use hpyerlinks
\usepackage{nomencl}
\makenomenclature
\usepackage{etoolbox}
\renewcommand\nomgroup[1]{%
	\item[\bfseries
	\ifstrequal{#1}{P}{Physics constants}{%
		\ifstrequal{#1}{N}{Number sets}{%
			\ifstrequal{#1}{O}{Other symbols}{}}}%
	]}
\usepackage{subcaption} 





%% Defs
\renewcommand\Affilfont{\fontsize{9}{10.8}\itshape}
\renewcommand{\d}{\text{d}}

\newcommand{\tr}{\text{tr}}
\newcommand{\der}[2]{\dfrac{\text{d} #1}{\text{d} #2}}
\newcommand{\pder}[2]{\dfrac{\partial #1}{\partial #2}}
\newcommand{\vavg}[1]{\left<#1\right>}
\newcommand{\sig}{\bm{\sigma}}
\newcommand{\bsig}{\bar{\bm{\sigma}}}
\newcommand{\hsig}{\hat{\bm{\sigma}}}
\newcommand{\eps}{\varepsilon}
\newcommand{\beps}{\bar{\bm{\varepsilon}}}
\newcommand{\heps}{\hat{\bm{\varepsilon}}}
\newcommand{\Eeff}{E_{\text{eff}}}
\newcommand{\red}[1]{{\color{red}#1}}
\newcommand{\abaqus}{{\sc Abaqus}}
\newcommand{\umat}{{\texttt{\sc umat}}}
\newcommand{\vumat}{{\texttt{\sc vumat}}}


\newcommand{\singlefig}[4]{
	\begin{figure}[!htb]
		\centering
		\includegraphics[width=#1\linewidth]{#2}
		\caption{#3}
		\label{fig:#4}
	\end{figure}
}

\newcommand{\doublefig}[4]{
	\begin{figure}[!htb]
		\centering
		\begin{subfigure}{0.48\textwidth}
			\centering 
			\includegraphics[width=\linewidth]{#1} 
			\caption{}
		\end{subfigure}\hfill
		\begin{subfigure}{0.48\textwidth}
			\includegraphics[width=\linewidth]{#2}
			\caption{}
		\end{subfigure}
		
		\caption{#3}
		\label{fig:#4}
	\end{figure}
}

\newcommand{\quadfig}[6]{
	\begin{figure}[!htb]
		\centering
		\begin{subfigure}{0.48\textwidth}
			\centering 
			\includegraphics[width=\linewidth]{#1} 
			\caption{}
		\end{subfigure}\hfil
		\begin{subfigure}{0.48\textwidth}
			\includegraphics[width=\linewidth]{#2}
			\caption{}
		\end{subfigure}
		
		\begin{subfigure}{0.48\textwidth}
			\centering 
			\includegraphics[width=\linewidth]{#3} 
			\caption{}
		\end{subfigure}\hfil
		\begin{subfigure}{0.48\textwidth}
			\includegraphics[width=\linewidth]{#4}
			\caption{}
		\end{subfigure}
		
		\caption{#5}
		\label{fig:#6}
	\end{figure}
}


\newcommand{\quinfig}[8]{
	\begin{figure}[!htb]
		\centering
		\begin{subfigure}{#8\textwidth}
			\centering 
			\includegraphics[width=\linewidth]{#1} 
			\caption{}
		\end{subfigure}\hfil
		\begin{subfigure}{#8\textwidth}
			\includegraphics[width=\linewidth]{#2}
			\caption{}
		\end{subfigure}
		
		\begin{subfigure}{#8\textwidth}
			\centering 
			\includegraphics[width=\linewidth]{#3} 
			\caption{}
		\end{subfigure}\hfil
		\begin{subfigure}{#8\textwidth}
			\includegraphics[width=\linewidth]{#4}
			\caption{}
		\end{subfigure}
		
		\begin{subfigure}{#8\textwidth}
			\includegraphics[width=\linewidth]{#5}
			\caption{}
		\end{subfigure}
		
		\caption{#6}
		\label{fig:#7}
	\end{figure}
}

\newcommand{\hexfig}[8]{
	\begin{figure}[!htb]
		\centering
		\begin{subfigure}{0.31\textwidth}
			\centering 
			\includegraphics[width=\linewidth]{#1} 
			\caption{}
		\end{subfigure}\hfil
		\begin{subfigure}{0.31\textwidth}
			\includegraphics[width=\linewidth]{#2}
			\caption{}
		\end{subfigure}\hfil
		\begin{subfigure}{0.31\textwidth}
			\includegraphics[width=\linewidth]{#3}
			\caption{}
		\end{subfigure}
		
		\begin{subfigure}{0.31\textwidth}
			\centering 
			\includegraphics[width=\linewidth]{#4} 
			\caption{}
		\end{subfigure}\hfil
		\begin{subfigure}{0.31\textwidth}
			\includegraphics[width=\linewidth]{#5}
			\caption{}
		\end{subfigure}\hfil
		\begin{subfigure}{0.31\textwidth}
			\includegraphics[width=\linewidth]{#6}
			\caption{}
		\end{subfigure}
		
		\caption{#7}
		\label{fig:#8}
	\end{figure}
}


\newcommand{\heptfig}[9]{
	\begin{figure}[!htb]
		\centering
		\begin{subfigure}{0.31\textwidth}
			\centering 
			\includegraphics[width=\linewidth]{#1} 
			\caption{}
		\end{subfigure}\hfil
		\begin{subfigure}{0.31\textwidth}
			\includegraphics[width=\linewidth]{#2}
			\caption{}
		\end{subfigure}\hfil
		\begin{subfigure}{0.31\textwidth}
			\includegraphics[width=\linewidth]{#3}
			\caption{}
		\end{subfigure}
		
		\begin{subfigure}{0.31\textwidth}
			\centering 
			\includegraphics[width=\linewidth]{#4} 
			\caption{}
		\end{subfigure}\hfil
		\begin{subfigure}{0.31\textwidth}
			\includegraphics[width=\linewidth]{#5}
			\caption{}
		\end{subfigure}\hfil
		\begin{subfigure}{0.31\textwidth}
			\includegraphics[width=\linewidth]{#6}
			\caption{}
		\end{subfigure}
		
		
		\begin{subfigure}{0.32\textwidth}
			\includegraphics[width=\linewidth]{#7}
			\caption{}
		\end{subfigure}
		\caption{#8}
		\label{fig:#9}
	\end{figure}
}



\graphicspath{{./inkscape-figures/}{./lit-figures/}{./results-figures/}} 
\bibliography{piston-vs-taper.bib}

\title{Development of a machine learning model to determine the forces on the piston in the pump-tube of a two-stage gas gun deforming due to a taper}
\author[1, 2]{Benjamin Alheit\thanks{alhben001@myuct.ac.za}}
\date{\today}

\affil[1]{Centre for Research in Computational \& Applied Mechanics, University of Cape Town, 7701 Rondebosch, South Africa}
\affil[2]{Department of Mechanical Engineering, University of Cape Town, 7701 Rondebosch, South Africa}

\begin{document}
\maketitle
\begin{abstract}
\red{	Abstract here.}
\end{abstract}
\red{
\section{TODO}
\begin{enumerate}
	\item Get output parameters
	\item Model automation
	\item Mesh optimization
\end{enumerate}
}

\section{Introduction}
=
\red{
\begin{itemize}
	\item We will use: ton, mm, s, N, MPa, N-mm
\end{itemize}
}

\singlefig{0.7}{units}{Consistent units for \abaqus. From https://www.researchgate.net/post/What-are-the-Abaqus-Units-in-Visualization}{units}

\subsection{Gas gun design}

\subsubsection{title}

\section{Finite element model}

\subsection{Geometry}
\singlefig{0.99}{annotated-assembly}{Geometric model of two-stage gas gun pump tube.}{\label{fig:assembly}}
\subsection{Material models}
The material models used for each component are presented here.
\subsubsection{High density polyethylene (HDPE)}
High Density Polyethylene (HDPE) is modelled using the Ramberg-Osgood model as done so in the literature \cite{amjadi2020tensile}. In 1D, the model is given by
\begin{equation}
	E_{c}\varepsilon = \sigma + \alpha_{c} \sigma\left[\dfrac{|\sigma|}{\sigma_{yc}}\right]^{n-1}\,,
	\label{eq:ro-model}
\end{equation} 
where $E_{c}$ is the Young's modulus, $\alpha_{c}$ is the yield offset, $\sigma_{yc}$ is the yield stress and $n$ is the hardening exponent. In 3D, the model has the additional parameter of Poisson's ratio, denoted by $\nu_{c}$. The material parameters are determined by fitting the model to data from \cite{amjadi2020tensile} using a \texttt{Python} script which can be found at \bhref{https://github.com/BenAlheit/piston-vs-taper/blob/master/computation/determining_material_parameters/hdpe_room_temp.py}{this link}. Additionally, a 3D uniaxial \abaqus simulation utilizing the model is compared to the 1D model (equation \eqref{eq:ro-model}) to verify that the model behaves as expected in \abaqus. The results of the simulation, the behaviour of the model, and the data are displayed in Figure \ref{fig:stress-strain}.
\singlefig{0.7}{hdpe-stress-strain-params}{Stress-strain relation for HDPE modelled with the Ramberg-Osgood model. The model is calibrated to fit data from \cite{amjadi2020tensile} (denoted by green `x's). Additionally, the 1D model is compared to a uniaxial \abaqus simulation using the calibrated material parameters. It is clear that the curves match each other and provide a good fit to the data.}{\label{fig:stress-strain}}
The calibrated material parameters, as well as the other parameters required to define the model, are presented in Table \ref{tab:hdpe-mat-params}.
\begin{table}[!htb]
	\newcommand{\colsize}{2.cm}
	\centering
	\caption{Material parameters for HDPE}
	\label{tab:hdpe-mat-params}
	\begin{tabular}{p{\colsize}| p{\colsize} | p{\colsize} | p{\colsize} | p{\colsize} | p{\colsize}}
		Density $\rho_{c}$ ton/mm\textsubscript{3} \cite{dasari2003microstructural, amjadi2020tensile} & Young's modulus $E_{c}$ MPa & Poisson's ratio $\nu_{c}$ \cite{nitta2012poisson} &  Yield stress $\sigma_{yc}$ MPa & Yield offset $\alpha_{c}$ & Hardening exponent $n$\\
		\hline\hline
		$0.95\times 10^{-9}$ & $1.63\times 10^{3}$ & $0.46$ & $14.6$ & 0.342 & 4.26
	\end{tabular}
\end{table}


\subsubsection{Mild steel}
The mild steel components are expected to remain in the elastic region and so only linear elastic material behaviour is considered. The material parameters are given in Table \ref{tab:mild-steel-mat-params}.
\begin{table}[!htb]
	\newcommand{\colsize}{2.cm}
	\centering
	\caption{Material parameters for mild steel \red{Ref later}}
	\label{tab:mild-steel-mat-params}
	\begin{tabular}{p{\colsize}| p{\colsize} | p{\colsize} }
		Density $\rho_{t}$ ton/mm\textsubscript{3}  & Young's modulus $E_{t}$ MPa & Poisson's ratio $\nu_{t}$ \\
		\hline\hline
	$8\times 10^{-9}$ & $200\times10^{3}$ & 0.3
	\end{tabular}
\end{table}



\subsubsection{Aluminium}
The aluminium components are expected to remain in the elastic region and so only linear elastic material behaviour is considered. The material parameters are given in Table \ref{tab:aluminium-mat-params}.
\begin{table}[!htb]
	\newcommand{\colsize}{2.cm}
	\centering
	\caption{Material parameters for aluminium taken from \cite{aluminium-props}}
	\label{tab:aluminium-mat-params}
	\begin{tabular}{p{\colsize}| p{\colsize} | p{\colsize} }
		Density $\rho_{p}$ ton/mm\textsubscript{3}  & Young's modulus $E_{p}$ MPa & Poisson's ratio $\nu_{p}$ \\
		\hline\hline
		$2.69\times 10^{-9}$ & $68.3\times10^{3}$ & 0.34
	\end{tabular}
\end{table}

\subsection{Interactions}
\subsubsection{Cap and tube}
\subsubsection{Cap and transition piece}
\subsubsection{Cap and piston}
\subsubsection{Piston and tube}
\subsubsection{Tube and transition piece}

\subsection{Initial and boundary conditions}
\subsection{Single test simulation}

\subsection{Mesh optimization}

Parameters
\begin{enumerate}
	\item ratio of element expansion for piston
	\item ratio of element expansion for tube
	\item n elements
\end{enumerate}

Objective function
\begin{enumerate}
	\item Min elements
\end{enumerate}

Constraint
\begin{enumerate}
	\item Force
	\item Dissipation
\end{enumerate}

\section{Machine learning surrogate model}
\subsection{Feature engineering}

\red{
Predictive features:
\begin{enumerate}
	\item Coefficient of friction: $\mu$
	\item Taper angle: $\alpha$
	\item Velocity: $v$
	\item Distance between piston front and taper start: $x_{taper}$
	\item Pressure difference between piston front and back: $\Delta p$
	\item Piston length: $l_{p}$
	\item Piston density: $\rho_{p}$
	\item Accumulative plastic strain in the piston: $\gamma$ 
\end{enumerate}

Dependent variables:
\begin{enumerate}
	\item Axial force on piston due to taper: $F_{z}$
	\item Increment in accumulated plastic dissipation: $\Delta \gamma$ 
\end{enumerate}
}

\subsubsection{Dimensional analysis}

\begin{table}[!htb]
	\centering
	\caption{Fundamental dimensional units}
	\label{tab:dimentional-units}
	\begin{tabular}{c c c}
		\hline\hline
		Mass $[M]$ & Length $[L]$ & Time $[T]$\\
		\hline\hline
	\end{tabular}
\end{table}

\begin{table}[!htb]
	\centering
	\caption{Variables on which the problem depends \red{Check that units are correct}}
	\label{tab:dimentional units}
	\begin{tabular}{l p{7cm} c c}
		\hline
		Component & Description & Symbol & Units \\
		\hline \hline
		Cap & Axial contact force & $F_{c}$ & $[MLT^{-2}]$ \\
		&Rate of volume average plastic dissipation & $\vavg{\dot{\Phi}_{c}}$ & $[ML^{-1}T^{-3}]$\\
		&Volume average plastic dissipation & $\vavg{\Phi_{c}}$ & $[ML^{-1}T^{-2}]$\\
		& Pressure difference & $\Delta P$ & $[ML^{-1}T^{-2}]$\\
		& Velocity & $v$ & $[LT^{-1}]$\\
		&Distance of cap past taper & $x_{t}$ & $[L]$\\
		&Length past piston & $l_{c}$ & $[L]$\\
		&Diameter & $d_{c}$ & $[L]$\\
		&Density & $\rho_{c}$ & $[ML^{-3}]$\\
		&Young's modulus & $E_{c}$ & $[ML^{-1}T^{-2}]$\\
		&Poisson's ratio & $\nu_{c}$ & $[\bullet]$\\
		&Yield stress & $\sigma_{yc}$ & $[ML^{-1}T^{-2}]$\\
		\hline
		Piston&Length & $l_{p}$ & $[L]$\\
		&Diameter & $d_{p}$ & $[L]$\\
		&Density & $\rho_{p}$ & $[ML^{-3}]$\\
		&Young's modulus & $E_{p}$ & $[ML^{-1}T^{-2}]$\\
		&Poisson's ratio & $\nu_{p}$ & $[\bullet]$\\
		\hline
		Transition piece& Length & $l_{t}$ & $[L]$\\
		& Diameter change & $\Delta d_{t}$ & $[L]$\\
		&Density & $\rho_{t}$ & $[ML^{-3}]$\\
		&Young's modulus & $E_{t}$ & $[ML^{-1}T^{-2}]$\\
		&Poisson's ratio & $\nu_{t}$ & $[\bullet]$\\
		\hline
		Friction & Cap-on-steel CoF & $\mu_{cs}$ & $[\bullet]$\\	
		&Steel-on-steel CoF & $\mu_{ss}$ & $[\bullet]$\\
		&Aluminium-on-steel CoF & $\mu_{as}$ & $[\bullet]$\\	
	\end{tabular}
\end{table}

\begin{equation}
	\Pi = F_c^{k_{1}}\vavg{\dot{\Phi}_{c}}^{k_{2}}\Delta P^{k_{4}}v^{k_{5}}\vavg{\Phi_{c}}^{k_{3}}x_{t}^{k_{6}}
\end{equation}
\begin{equation}
	r=3
\end{equation}
\begin{equation}
	n=24
\end{equation}
\begin{equation}
	n_{\text{n-dim}} = n-r=21
\end{equation}

\begin{equation}
	\newcommand{\dimmatspace}{\,\,\,\,\,}
	\begin{matrix}
		& 
		\,\,\,\,\,\,_{F_c} \dimmatspace _{\vavg{\dot{\Phi}_{c}}} \dimmatspace _{\Delta P} \,\,\dimmatspace _{v} \,\, \dimmatspace _{\vavg{\Phi_{c}}} \dimmatspace _{x_{t}} \dimmatspace
		\\
		\begin{matrix}
			_M \\ _L \\ _T 
		\end{matrix} & \left[\begin{matrix}
			1  &  1 &  1 &  0 &  1 & 0\\ 
			1  & -1 &  -1 &  1 & -1 & 1\\ 
			-2 & -3 & -2 & -1 & -2 & 0\\ 
		\end{matrix}\right]
	\end{matrix} 	\left[\begin{matrix}
	k_{1} \\  k_{2} \\ k_{3} \\ k_{4} \\ k_{5} \\ k_{6} 
\end{matrix}\right] = \bm{0}\,.
\end{equation}
\red{
\begin{enumerate}
	\item Enter units as symbols in sympy
	\item Construct dimensional matrix
	\item Find solutions $k^{j}_i$ to dimensional matrix by $n-r$ values in $k$ to 0 (except for 1 which on sets to $1$).
	\item Automatically print out resulting dimensionless products table.
\end{enumerate}
}
\begin{table}
	\centering
	\caption{Non-dimensional parameters}
	\label{tab:non-dim}
	\begin{tabular}{c c c c c c c}
		& $F_c$ & $\vavg{\dot{\Phi}_{c}}$ & $\vavg{\Phi_{c}}$ & $\Delta P$ & $v$ & $x_{t}$ \\
		& $k_{1}$ & $k_{2}$ & $k_{3}$& $k_{4}$& $k_{5}$& $k_{6}$ \\
		
	\end{tabular}
\end{table}

\red{
Based on \cite{hutter2013continuum}:

Notes:
\begin{enumerate}
	\item 
\end{enumerate}
Direct quote from \cite{hutter2013continuum}
\begin{quote}
	``Dimensional analysis is a method with the aid of which one may for instance test a formula for dimensional correctness. It leads to a first understanding of the solution of a physical problem and yields a precise information about the number of variables that are necessary to describe it, a fact that is particularly important when experiments are being performed. Very often dimensional analysis reduces the number of variables upon which a physical problem was initially surmised to depend. If for instance the quantity $y$ depends upon $x_{1}, x_{2} , ... , x_{n}$, where all quantities have a certain physical dimension, then dimensional analysis shows that y can only depend upon certain products of powers of $x_{1}, x_{2} , ... , x_{n}$, a fact that corresponds regularly to a considerable reduction of the number of variables. Naturally then, experiments may more simply or more economically be performed than without knowledge of this fact.''
\end{quote}

\begin{quote}
	``The first step in a dimensional analysis consists in the listing of the parameters, which influence a physical problem. This step is very decisive. If too many variables are listed that may describe a physical problem, then the final equations will contain superfluous variables, if too few variables are introduced, incomplete equations may emerge, which results in incomplete equations or "more often" false inferences or the result can not be expressed in terms of dimensionally homogeneous functions.''
\end{quote}
}

\subsection{Experimental input parameters}
 
In order of importance:
\begin{enumerate}
	\item Initial velocity: $v_{0}$
	\item Pressure path: $p_{path}$
	\item Piston length: $l_{p}$
	\item Coefficient of friction: $\mu$
\end{enumerate}

\subsection{Experimental results}

\subsection{Model}

\section{Packaging of model for use in 1D code}

\subsection{PIP}

\subsection{Usage example}


	\printbibliography
\end{document}